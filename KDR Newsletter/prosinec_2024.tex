\documentclass{newsletter}

\graphicspath{ {./obrazky/} }

\kdrtitle{Prosinec 2024}
\kdrsubtitle{Newsletter komise rozhodčích a delegátů}
\kdrbackground{listopad-2024-background.jpg}

\begin{document}

\clanek{Vážení rozhodčí a delegáti,}{}
Dovolte mi vás přivítat u listopadového newsletteru od KRD USaLB. Omlouváme se za delší prodlevu od předchozího vydání, ovšem věříme, že o to víc je tento měsíc nadupanější. Připravili jsme si pro vás pestrou škálu témat, která se týkají pravidel a prací se elektronickým zápisem o utkání (EZOU), ve které došlo nedávno ke změnám. Věříme, že vás jednotlivá témata zaujmou a pomohou Vám lépe zvládat specifické situace na hřišti i mimo něj.

Tento měsíc jsme si pro Vás připravili newsletter s následujícími tématy:
\begin{itemize}
	\item Výklad pravidel o nesprávném vybavení (florbalka)
	\item Odmítání soubojů a aplikace pravidel
	\item Výklad pravidel o brance z tečované čepele
	\item Novinky a semináře v lednu
\end{itemize}

KRD bude opět ráda za jakoukoliv zpětnou vazbu, která nám pomůže vylepšit newslettery do budoucna tak, aby Vám co nejlépe pomáhali při činnosti rozhodčích a delegátů.

S přátelským pozdravem,

\begin{flushleft}
	\vspace{3\baselineskip}
	\includegraphics[width=3.5cm, keepaspectratio]{tadeas_sidenberg_podpis}\\
	\textbf{Tadeáš Sidenberg}\\
	předseda KRD USaLB
\end{flushleft}

\pagebreak
\clanek{Nesprávné vybavení}{Výklad pravidel a jejich aplikace}
KRD tímto připomíná pravidla týkající se kontroly hráčského vybavení, a to především florbalek. Kontrolu osobního vybavení provádějí rozhodčí, avšak kapitáni mají právo o kontrolu požádat. V případě florbalek a brankářských masek je podle pravidla 409/2 povinností rozhodčích této žádosti vyhovět. Při kontrole florbalky můžete určit, zda spadá do jedné z následujících kategorií:

\begin{itemize}
	\item \textbf{Neschválená florbalka}: Uděluje se \textcolor{cfred}{\textbf{technický trest}}. Za neschválenou florbalku se považují situace, kdy chybí schvalovací známka, hůl a čepel nejsou od stejného výrobce nebo zahnutí čepele přesahuje 30 mm.
	\item \textbf{Upravená florbalka}: Uděluje se \textcolor{cfred}{\textbf{trest do konce utkání}}. Za upravenou florbalku se považují situace, kdy je florbalka prodloužená nebo zpevněná.
	\item \textbf{Vadná florbalka}: Hráč je upozorněn a \textbf{vyzván k nápravě}. Pokud ji neprovede a pokračuje s ní ve hře, uděluje se \textcolor{cfred}{\textbf{trest do konce utkání}}. Za vadnou florbalku se považují například situace, kdy je hůl nebo čepel poškozená, či spojení mezi holí a čepelí není pevné.
	\item \textbf{Schválená florbalka bez závad}: Hráč pokračuje ve hře. Pokud kontrolu vyžádal kapitán při přerušení, je potrestán 2 minutami za zdržování hry.
\end{itemize}

Postup měření zahnutí čepele je pevně definován v pravidle 406/2, kdy se čepel florbalky položí na rovný povrch a změří se vzdálenost od povrchu k nejvyššímu bodu na spodním oblouku florbalky.

\begin{pravidlo}
	{406 Florbalka}
	{Čepel nesmí být ostrá, zahnutí čepele nesmí přesáhnout 30 mm}
	{2} Všechny úpravy čepele, kromě ohnutí jsou zakázány. \textcolor{cfred}{Zahnutí čepele se měří jako vzdálenost od spodního okraje nejvyššího bodu vnitřní strany čepele a rovného povrchu, na kterém florbalka leží}. Výměna čepele je povolena, pokud je čepel schválená s holí a jsou stejné značky, ale nová čepel nesmí být zeslabena. Polepení spojnice mezi čepelí a holí je dovoleno, ale maximálně 10 mm viditelné části čepele smí být takto překryto.
\end{pravidlo}


\clearpage
\clanek{Odmítání soubojů}{Výklad pravidel a jejich aplikace}

\clearpage
\clanek{Gól z teče o čepel}{Výklad pravidel a jejich aplikace}

Smysl pravidla je podle pravidlové komise následující:

\begin{itemize}
	\item Pokud hráč v poli docílí branky florbalkou, pak branka má platit.
	\item Pokud hráč v poli docílí branky nohou, pak branka nemá platit.
\end{itemize}

\clearpage
\clanek{Novinky a semináře}{Na prosinec 2024}

\begin{pravidlo}
	{409 Kontrola výstroje}
	{Rozhodčí rozhodují o kontrole a přeměřování veškeré výstroje.}
	{1} Kontrola probíhá před a během utkání. Nesprávná výstroj, včetně vadné florbalky, vyjma měření zahnutí, objevená
	před či během utkání, musí být uvedena do řádného stavu příslušným hráčem, který poté může nastoupit k utkání
	či v něm pokračovat. Přestupky týkající se dresů hráčů nevedou k více než jednomu trestu pro družstvo za utkání. Jakékoliv nesprávné vybavení musí být zaznamenáno do zápisu o utkání.
\end{pravidlo}

Ani na konci roku nezastavujeme a první dva týdny adventu máme pro každého něco. Pro nováčky máme přichystaný seminář na emoce a tresty, kde se dozvíte o základech řízení utkání a zvládání nejenom emocí hráčů, ale především těch vlastních. Pro pokročilejší rozhodčí jsme si nachystali hned dva semináře zaměřené na vnímání utkání a osobnost rozhodčího, což jsou nedílné součásti vaší profese. Hned ze začátku měsíce máme připravený seminář na aplikaci pravidel 2, kde se podíváme na aplikaci pravidel při hře florbalkou.

\begin{table}[h]
	\centering
	\renewcommand{\arraystretch}{2}
	\begin{tabular}{| r | l | l | l |}
		\hline
		02. 12. 2024 & Aplikace pravidel 2 & Licence D & online teams ÚsaLb \\
		\hline
		03. 12. 2024 & Vnímání utkání & Licence C & online teams ÚsaLb \\
		\hline
		11. 12. 2024 & Emoce 1 + Tresty 1 & Licence E & Sportovní Areály Most \\
		\hline
		12. 12. 2024 & Osobnost rozhodčího & Licence C & online teams ÚsaLb \\
		\hline
	\end{tabular}
\end{table}

\end{document}