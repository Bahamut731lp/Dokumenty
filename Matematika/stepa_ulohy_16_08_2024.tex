\documentclass[]{article}
\usepackage[a4paper, total={6in, 8in}]{geometry}
\usepackage[parfill]{parskip}
\usepackage{titlesec}

\titleformat{\section}[display]
{\normalfont\normalsize\bfseries\centering}{Sekce \thesection.}{-0.5em}{}
\titleformat{\subsection}
{\normalfont\normalsize\itshape}{\thesubsection.}{1em}{}
\titleformat{\subsubsection}
{\normalfont\normalsize\itshape}{\thesubsubsection.}{1em}{}

\renewcommand\thesection{\Roman{section}}
\renewcommand\thesubsection{\thesection.\Roman{subsection}}

\setcounter{secnumdepth}{5}

%opening
\title{
	{\textbf{Úlohy na pátek 16.8.2024}}
}

\date{}

\begin{document}
	
	\maketitle
	
	\section{Zadání}
	
	\begin{enumerate}
		\item Určete $a \cdot b$, je-li dána velikost vektorů a jejich úhlu $\varphi$. $|a| = 2$, $|b|=3$, $\varphi = 90^\circ$.
		\item Určete velikost úhlu mezi vektory $a$, $b$, je-li dána velikost $|a| = 2$, $|b| = 3$, $a \cdot b = -3\cdot\sqrt{3}$ 
		\item Určete chybějící souřadnici vektoru $u = (2, u_2)$ tak, aby byl kolmý k vektoru $v = (1, 2)$.
		\item Určete vektor $u$ tak, aby měl velikost 10 a přitom byl kolmý k danému vektoru $v = (-1, 2)$.
		\item Jsou dány vrcholy trojúhelníku ABC. Určete velikosti jeho vnitřních úhlů a obsah. $A[1, 0], B[2,0], C[2, \sqrt{3}]$.
		\item Vypočtěte souřadnice vektorového součinu $u \times v$, je-li
		\begin{enumerate}
			\item $u = (2, -1, 3), v = (3, 2, -2)$
			\item $u = (-4, -6, 0), v = (2, -7, 0)$
			\item $u = (1, -2, 3), v = (-2, 4, -6)$
		\end{enumerate}
		\item Jsou dány body $A=[0,1,3], B[2,0,-1], C[1,-2,0]$. Určete souřadnice normálového vektoru roviny ABC.
	\end{enumerate}
	
	\pagebreak
	\section{Nápovědy}
	
	\begin{enumerate}
		\item Určete $a \cdot b$, je-li dána velikost vektorů a jejich úhlu $\varphi$. $|a| = 2$, $|b|=3$, $\varphi = 90^\circ$.
		\begin{enumerate}
			\item Jaké znáš vzorečky skalárního součinu vektorů?
			\item Můžeme si nějak $a \cdot b$ vyjádřit ze vzorečku skalárního součinu?
		\end{enumerate}
		\item Určete velikost úhlu mezi vektory $a$, $b$, je-li dána velikost $|a| = 2$, $|b| = 3$, $a \cdot b = -3\cdot\sqrt{3}$ 
		\begin{enumerate}
			\item Opět si připomeň vzoreček skalárního součinu
			\item Literally jenom dosaď.
		\end{enumerate}
		\item Určete chybějící souřadnici vektoru $u = (2, u_2)$ tak, aby byl kolmý k vektoru $v = (1, 2)$
		\begin{enumerate}
			\item Kolmé vektory nemusíme dlouhosáhle počítat - jaká je finta na jeho vytvoření?
			\item První souřadnice $u$ je 2 - vynásob si $v$ dvojkou, aby bylo vidět, co za číslo tam hodit.
		\end{enumerate}
		\item Určete vektor $u$ tak, aby měl velikost 10 a přitom byl kolmý k danému vektoru $v = (-1, 2)$.
		\begin{enumerate}
			\item Stejně jako v předchozí úloze - najít kolmý vektor pomocí otočení znaménka jedné ze souřadnic.
			\item Velikost závisí na souřadnicích. Pokud má mít velikost 10, musí ti pod odmocninou vyjít 100.
		\end{enumerate}		
		\item Jsou dány vrcholy trojúhelníku ABC. Určete velikosti jeho vnitřních úhlů a obsah. $A[1, 0], B[2,0], C[2, \sqrt{3}]$.
		\begin{enumerate}
			\item Z bodů si vytvoř vektory, které budou reprezentovat jednotlivé strany trojúhelníka.
			\item Velikost vnitřních úhlů spočítáš jako úhel mezi vektory stran.
			\item Velikost úhlu mezi vektory spočítáš pomocí skalárního součinu.
		\end{enumerate}
		\item Vypočtěte souřadnice vektorového součinu $u \times v$, je-li
		\begin{enumerate}
			\item Najdi si vzoreček pro výpočet vektorového součinu.
		\end{enumerate}
		\item Jsou dány body $A=[0,1,3], B[2,0,-1], C[1,-2,0]$. Určete souřadnice normálového vektoru roviny ABC.
		\begin{enumerate}
			\item Normálový znamená kolmý.
			\item Vypočítat kolmý vektor k $n$ dalším vektorům lze pomocí vektorového součinu.
		\end{enumerate}
	\end{enumerate}
\end{document}
