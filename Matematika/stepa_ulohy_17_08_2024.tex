\documentclass[]{article}
\usepackage[a4paper, total={6in, 8in}]{geometry}
\usepackage[parfill]{parskip}
\usepackage{titlesec}

\titleformat{\section}[display]
{\normalfont\normalsize\bfseries\centering}{Sekce \thesection.}{-0.5em}{}
\titleformat{\subsection}
{\normalfont\normalsize\itshape}{\thesubsection.}{1em}{}
\titleformat{\subsubsection}
{\normalfont\normalsize\itshape}{\thesubsubsection.}{1em}{}

\renewcommand\thesection{\Roman{section}}
\renewcommand\thesubsection{\thesection.\Roman{subsection}}

\setcounter{secnumdepth}{5}

%opening
\title{
	{\textbf{Úlohy na sobotu 17.8.2024}}
}

\date{}

\begin{document}
	
	\maketitle
	
	\section{Zadání}
	
	\begin{enumerate}
		\item Napiště parametrické vyjádření přímky určené bodem $A$ a směrovým vektorem $a$.
		\begin{enumerate}
			\item $A[3, -7], a = (2, -1)$
			\item $A[4,0], a = (0,5)$
		\end{enumerate}
		\item Napiště parametrické vyjádření přímky $AB$, polopřímky $AB$, úsečky $AB$ a polopřímky $BA$, jsou-li dány souřadnice bodů $A$, $B$.
		\begin{enumerate}
			\item $A[2, -7], B[-3, 1]$
			\item $A[3, -1], B[-2, -1]$
		\end{enumerate} 
		\item Sestavte obecnou rovnici přímky, která je určena
		\begin{enumerate}
			\item bodem $A[-3, 2]$ a normálovým vektorem $n=(2,1)$
			\item bodem $A[-3, 2]$ a směrovým vektorem $s=(3,-2)$
			\item body $A[-3, 2]$ a $A[-2, 4]$
			\item parametrickým vyjádřením: $x = 2 - t, y = -3 + 2t$
		\end{enumerate}
		
		\item Je dáno parametrické vyjádření přímky $p: x = -2 + t, y = 2 - 2t$ a body $A[2, -6], B[0,4], C[3, c_2]$.
		\begin{enumerate}
			\item Rozhodněte, který z bodů $A, B$ leží na přímce $p$.
			\item Určete chybějící souřadnici $c_2$ tak, aby $C \in p$.
			\item Určete průsečíky přímky $p$ s osami $x$ a $y$.
		\end{enumerate}
	\end{enumerate}
	
	\pagebreak
	\section{Nápovědy}
	
	\begin{enumerate}
		\item Napiště parametrické vyjádření přímky určené bodem $A$ a směrovým vektorem $a$.
		\begin{enumerate}
			\item Parametrické vyjádření přímky $p$ je jako $p: X = A + a\cdot t$, kde $A$ je bod a $a$ je směrový vektor.
			\item Parametrické vyjádření přímky lze napsat vektorově jako jednu rovnici, nebo pro každou souřadnici zvlášť (co souřadnice to rovnice).
		\end{enumerate}
		\item Napiště parametrické vyjádření přímky $AB$, polopřímky $AB$, úsečky $AB$ a polopřímky $BA$, jsou-li dány souřadnice bodů $A$, $B$.
		\begin{enumerate}
			\item Polopřímka $AB$ je přímka, která začíná v bodě $A$ a jde přes bod $B$ až do nekonečna. Naopak polopřímka $BA$ začíná v bodě $B$ a jde přes bod $A$ až do nekonečna.
			\item Úloha spočívá v omezení parametru $t$.
			\item Když má být vyjádřena polopřímka, nesmí být parametr záporný (kdyby to bylo možné, tak by se vektor otočil za počáteční bod)
			\item Když má být vyjádřena úsečka, tak směrový vektor může být jenom v rozsahu $\left<0, 1\right>$. Kdyby byl záporný, tak jde na druhou stranu, a kdyby byl větší než 1, tak úsečku "přeroste" ven.
		\end{enumerate} 
		\item Sestavte obecnou rovnici přímky, která je určena
		\begin{enumerate}
			\item Obecná rovnice přímky je $p: ax + by + c = 0$, kde $a, b$ jsou souřadnice normálového vektoru, a $x, y$ souřadnice výchozího bodu. Nezapomeň z toho dopočítat parametr $c$.
			\item Směrový vektor je kolmý na vektor normálový.
			\item Vytvoř směrový vektor mezi těmito dvěma body
			\item Vytahej si z předpisu parametrické přímky směrový vektor a bod, a postupuj jako v druhé podúloze.
		\end{enumerate}
		
		\item Je dáno parametrické vyjádření přímky $p: x = -2 + t, y = 2 - 2t$ a body $A[2, -6], B[0,4], C[3, c_2]$.
		\begin{enumerate}
			\item Stačí dosadit souřadnice do rovnic a kouknout, jestli je ve všech rovnicích stejný parametr $t$.
			\item Stačí dosadit do rovnice pro $x$-ovou souřadnici trojku, spočítat $t$ a dosadit do rovnice pro $y$.
			\item Zde se nedosazují souřadnice bodu, ale nuly. Nejdříve jenom za $x$, a potom jenom za $y$.
		\end{enumerate}
	\end{enumerate}
\end{document}
