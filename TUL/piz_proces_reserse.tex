\documentclass[FM]{tulpresentation}

\usepackage{hyperref}
\title{Jak dělám rešerši}
\subtitle{Aneb chaotická organizace chaosu}
\author{Kevin Daněk}

\begin{document}
	
	\TULtitleframe

	\begin{frame}
		\frametitle{Motivace}
		\begin{items}
			\item Nerad čtu nezajímavé texty.
			\item Rád vidím výstupy \textbf{rychle}.
			\item Předtím, než dočtu první článek, tak už mě to nebaví.
			\item Řešení? \textbf{Číst jenom to, co potřebuju}.
		\end{items}
	\end{frame}

	\begin{frame}
		\frametitle{Metodika}
	    \begin{items}
	      \item První krok je \textbf{vytvoření myšlenkové mapy}
	      \begin{subitems}
	      	\item Vytvořit uzel
	      	\item Najít související témata
	      	\item Přidat související témata do myšlenkové mapy
	      	\item \textit{Rinse and repeat}
	      \end{subitems}
	      \item Můžeme použít metodu tužka papír, whiteboard nebo \href{https://mermaid.live}{\texttt{mermaid.live}}
	    \end{items}
	\end{frame}
	
	\begin{frame}
		\frametitle{Metodika}
		\begin{items}
			\item S dostatečným počtem uzlů přichází \textbf{vytvoření osnovy}.
			\begin{subitems}
				\item Uzly myšlenkové mapy se uspořádají do chronologické posloupnosti
				\item Občas je nutné přidat nové naduzly \textit{(kapitoly)}.
			\end{subitems}
			\item Práce by se měla číst podobnou flow jako vyprávění
			\begin{subitems}
				\item Postupně přivádíme postavy \textit{(témata)} do příběhu
				\item Nejdříve \textit{world-building}, pak až samotný \textit{plot}.
			\end{subitems}
		\end{items}
	\end{frame}
	
	\begin{frame}
		\frametitle{Metodika}
		\begin{items}
			\item S první verzí osnovy vzniká \textbf{nástřel}.
			\item Začneme postupně psát tělo práce.
			\item Během psaní iterujeme nad myšlenkovou mapou a osnovou.
			\begin{subitems}
				\item Některá témata budeme muset přidat pro vysvětlení
				\item Některá témata se ukážou jako nadbytečná.
				\item Některá témata bude lepší posunout kvůli předpokladům o znalostech čtenáře.
			\end{subitems}
		\end{items}
	\end{frame}
	
	\begin{frame}
		\frametitle{Výstup}
		\begin{items}
			\item Výstupem je text, který:
			\begin{subitems}
				\item Se věnuje tématu (kořenový uzel)
				\item Buduje téma od nejjednoduších částí po tu nesložitější
				\item Buduje slovník a principy, které podpírají samotné téma
			\end{subitems}
		\end{items}
	\end{frame}
	
	\begin{frame}
		\frametitle{Uplatnění}
		\begin{items}
			\item Text nemusí být jenom základ pro odbornou práci.
			\item Vznikne ve své podstatě učebnice pro nás samotné.
			\begin{subitems}
				\item Téma je možné si kdykoliv připomenout a naučit se prakticky od znova.
				\item Pro lidi jako já k nezaplacení.
			\end{subitems}
			\item \href{https://xkcd.com/1205/}{\texttt{Done is better than perfect.}}
		\end{items}
	\end{frame}
	
	\begin{frame}
		\frametitle{FAQ}
		\begin{items}
			\item \textbf{Jak mám udělat myšlenkovou mapu, když o tématu nic nevím?}
			\begin{subitems}
				\item Hledejte. Google, ChatGPT, Deepseek, Blackbox, NotebookLM, cokoliv.
				\item Co najdeme, přidáme do myšlenkové mapy. To nás totiž donutí se zamyslet, jak to souvisí.
				\item Je to iterativní proces - nemusíte to všechno připravit na první dobrou.
			\end{subitems}
		\end{items}
	\end{frame}
	
	\begin{frame}
		\frametitle{FAQ}
		\begin{items}
			\item \textbf{Kdy už toho mám dost na to, abych začal dělat osnovu?}
			\begin{subitems}
				\item \textit{Rule of thumb} je takový, že v moment, co už jsem se dostal na uzly, která nejdou dále rozložit.
				\item Na listech té naší mapy totiž budou buďto ty nejjednoduší stavební prvky tématu, a nebo to, čemu se nechceme věnovat a musíte to v textu podpořit jinak.
				\item Od nejzákladnější stavebních bloků se totiž chronologicky dobudujeme až k původnímu tématu.
			\end{subitems}
		\end{items}
	\end{frame}
	
	\begin{frame}
		\frametitle{FAQ}
		\begin{items}
			\item \textbf{Co když se už dopracuji na původní téma? Ta práce potom končí?}
			\begin{subitems}
				\item Určitě ne, stejně jako jsme šli k jednoduším věcem, můžeme jít i k těm složitějším, které na tématu staví.
				\item Je dobré mít téma podepřené jeho základy, ale poté se již můžeme věnovat podobným, nebo složitějším věcem, protože k nim máme kontext.
			\end{subitems}
		\end{items}
	\end{frame}
  \TULendframe
\end{document}
