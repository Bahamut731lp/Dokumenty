\documentclass[a4paper,12pt,twoside,FP]{article}
% tento dokument používá balíky specifické pro XeLaTeX a lze jej přeložit
% jen XeLaTeXem

\newcommand{\verze}{2.3}

\usepackage{polyglossia}
\setdefaultlanguage{czech}

\usepackage{fontspec}
\usepackage{xunicode}
\usepackage{float}

\usepackage{xltxtra}
\usepackage{tabularx}

\usepackage{makeidx}
\makeindex

% živé odkazy v PDF
\usepackage{hyperref}
\hypersetup{colorlinks=true, linkcolor=tul, urlcolor=tul, citecolor=tul}
\hypersetup{pdftitle={Shrnutí mikroprojektů v euroregionu Silesia}}

\usepackage[fonts,noheader]{tul}
\TULname{Pavel Satrapa}
\TULphone{+420\,485\,351\,234}
\TULmail{Pavel.Satrapa@tul.cz}
%\TULposition{prorektor pro informatiku}

\pagestyle{TULheader}

% definice příkazů a prostředí specifických pro tento dokument
\newcommand{\cmdfont}[1]{\texttt{\color{\tulcolor}#1}}
\newcommand{\cmdnoindex}[1]{\cmdfont{\textbackslash #1}}
\newcommand{\cmd}[1]{\cmdnoindex{#1}\index{#1@\textbackslash #1}}
\newcommand{\demobox}{\raisebox{-.20ex}{\rule{1em}{1em}}}

\usepackage{enumitem}
\newlist{itemize*}{itemize}{2}
\setlist[itemize*]{itemsep=3pt, parsep=3pt, label=$\bullet$}
\setlist[itemize*,2]{label={--}}

\usepackage{parskip}
\sloppy
\widowpenalty=10000
\clubpenalty=10000

\newcommand{\cmddemo}[1]{\bigskip\parbox[c]{3.9cm}{\cmd{#1}}\parbox[c]{10cm}{\csname #1\endcsname}\bigskip}

\begin{document}

\TULfancytitlepage{Shrnutí mikroprojektů v euroregionu Silesia}
{Lucie Smrčková, Simona Šulcová, Aja Bohatá, Ondřej Málek, Kateřina Najmannová, Tereza Skálová}
{Liberec 2025}

\section*{Sběr dat o euroregionu}
Získávání informací o euroregionu Silesia bylo poměrně náročné a vyžadovalo systematický přístup. Nejprve jsme se snažili potřebná data najít na oficiálním webu euroregionu. Ačkoli zde bylo k dispozici mnoho informací, tak jejich rozložení na webových stránkách nám ztěžovalo jakékoliv hledání. Abychom proces urychlili, rozhodli jsme se kontaktovat kancelář euroregionu a zavolat panu manažerovu Bc. Radku Ivanovi. Po výměnně několika e-mailů jsme doufali v souhrnný dokument. Místo toho jsme však byli nasměrováni na zápisy jednání euroregionálního řídícího výboru, které obsahovaly přehledy projednávaných mikroprojektů. Tento seznam nám poskytl výchozí bod pro další práci.

Kolega Bc. Kevin Daněk z navazujícího magisterského studia aplikované informatiky nám pomohl s technickou částí analýzy dokumentů. Vysvětlil, že původní přehledy mikroprojektů byly organizovány podle jednotlivých jednání, na nichž byly schváleny, a rozděleny do tabulek ve formátu PDF podle typu projektu. Nejprve data manuálně převedl do textového formátu a data očistil od šumu. Následně stáhl a sloučil zápisy z jednání, aby mohl vytvořit jednotnou databázovou tabulku.

Po uložení těchto dat do databázového systému napsal SQL dotazy, pomocí nichž zjistil počet projektů jednotlivých žadatelů a partnerů. Díky této analýze bylo možné identifikovat nejčastější subjekty zapojené do mikroprojektů a získat přehled o celkovém rozložení projektů v euroregionu. Tento přístup byl podobný práci s kontingenčními tabulkami v běžném tabulkovém procesoru, avšak umožnil detailnější zpracování a přesnější analýzu.
\clearpage

\section*{Výsledky analýzy}
Na základě zpracovaných dat jsme provedli podrobnou analýzu mikroprojektů v euroregionu Silesia. Zaměřili jsme se na identifikaci nejaktivnějších subjektů z hlediska počtu podaných žádostí, nejčastěji zapojených partnerů a také na subjekty, které spolupracují nejvíce napříč projekty. Výsledky této analýzy jsou shrnuty v následujících tabulkách, které poskytují přehled o klíčových aktérech v rámci euroregionu a jejich vzájemných vazbách.

\bgroup
\begin{table}[H]
	\centering
	\renewcommand{\arraystretch}{1.25}%  1 is the default, change whatever you need
	\begin{tabular}{|l|l|}
		\hline
		\textbf{Žadatel} & \textbf{Počet projektů} \\
		\hline
		Gmina Kornowac & 30 \\
		\hline
		Miasto Racibórz & 21 \\
		\hline
		Gmina Krzyżanowice & 16 \\
		\hline
		Statutární město Ostrava & 14 \\
		\hline
		Miasto Wodzisław Śląski & 13 \\
		\hline
		Gmina Lyski & 12 \\
		\hline
	\end{tabular}
	
	\caption{Nejčastější žadatelů v euroregionu Silesia}
\end{table}
\egroup

\bgroup
\begin{table}[H]
	\centering
	\renewcommand{\arraystretch}{1.25}%  1 is the default, change whatever you need
	\begin{tabular}{|l|l|}
		\hline
		\textbf{Partner} & \textbf{Počet projektů} \\
		\hline
		Miasto Racibórz & 16 \\
		\hline
		Obec Šilheřovice & 14 \\
		\hline
		Obec Bolatice & 12 \\
		\hline
		Město Odry & 12 \\
		\hline
		Statutární město Ostrava & 11 \\
		\hline
		Obec Vřesina & 11 \\
		\hline
		Gmina Krzyżanowice & 11 \\
		\hline
		Gmina Kietrz & 10 \\
		\hline
	\end{tabular}
	
	\caption{Nejčastější partneři v euroregionu Silesia}
\end{table}
\egroup

\bgroup
\begin{table}[H]
	\centering
	\renewcommand{\arraystretch}{1.25}%  1 is the default, change whatever you need
	\begin{tabular}{|p{0.3\linewidth}|p{0.3\linewidth}|p{0.3\linewidth}|}
		\hline
		\textbf{Žadatel} & \textbf{Partner} & \textbf{Společné projekty} \\
		\hline
		Gmina Krzyżanowice & Obec Šilheřovice & 12 \\
		\hline
		Gmina Kornowac & Obec Vřesina & 10 \\
		\hline
		Gmina Lyski & Obec Darkovice & 9 \\
		\hline
		Miasto Wodzisław Śląski & Statutární město Ostrava & 7 \\
		\hline
		Stowarzyszenie LYSKOR & Obec Bolatice & 7 \\
		\hline
		Gmina Kornowac & Základní škola a Mateřská škola Opava - Komárov - příspěvková organizace & 6 \\
		\hline
	\end{tabular}
	
	\caption{Společné projekty subjektů v euroregionu Silesia}
\end{table}
\egroup

\bgroup
\begin{table}[H]
	\centering
	\renewcommand{\arraystretch}{1.25}%  1 is the default, change whatever you need
	
	\begin{tabular}{|c|c|}
		\hline
		\textbf{Tématický okruh} & \textbf{Počet projektů} \\
		\hline
		Turistika a příroda  & 66  \\
		\hline
		Historie, kultura a tradice & 114  \\
		\hline
		Sport a životní styl & 63  \\
		\hline
		Vzdělávání a věda & 64  \\
		\hline
		Bezpečnost, hasiči a krizové řízení & 25  \\
		\hline
		Společenská integrace a komunitní aktivity & 41  \\
		\hline
		Umění  & 39  \\
		\hline
		Ekologie  & 17  \\
		\hline
		Infrastruktura a regionální rozvoj & 21  \\
		\hline
		Gastronomie & 24  \\
		\hline
	\end{tabular}
	
	\caption{Počet projektů v tématických okruzích}
\end{table}
\egroup

\clearpage
\section*{Reference}
\begin{itemize}
	\item EUROREGION SILESIA. Euroregion Silesia. Online. Euroregion Silesia. Dostupné z: https://euroregion-silesia.cz/. [cit. 2025-03-03].
\end{itemize}

\end{document}

