\documentclass[a4paper,12pt,twoside,FM]{article}
% tento dokument používá balíky specifické pro XeLaTeX a lze jej přeložit
% jen XeLaTeXem

\newcommand{\verze}{2.3}

\usepackage{polyglossia}
\setdefaultlanguage{czech}

\usepackage{fontspec}
\usepackage{xunicode}

\usepackage{xltxtra}
\usepackage{tabularx}

\usepackage{makeidx}
\makeindex

% živé odkazy v PDF
\usepackage{hyperref}
\hypersetup{colorlinks=true, linkcolor=tul, urlcolor=tul, citecolor=tul}
\hypersetup{pdftitle={Návod k použití balíku tul pro LaTeX verze \verze}}

\usepackage[fonts, noheader]{tul}
\noTULfooter
\TULname{Pavel Satrapa}
\TULphone{+420\,485\,351\,234}
\TULmail{Pavel.Satrapa@tul.cz}
%\TULposition{prorektor pro informatiku}

% definice příkazů a prostředí specifických pro tento dokument
\newcommand{\cmdfont}[1]{\texttt{\color{\tulcolor}#1}}
\newcommand{\cmdnoindex}[1]{\cmdfont{\textbackslash #1}}
\newcommand{\cmd}[1]{\cmdnoindex{#1}\index{#1@\textbackslash #1}}
\newcommand{\demobox}{\raisebox{-.20ex}{\rule{1em}{1em}}}

\usepackage{enumitem}
\newlist{itemize*}{itemize}{2}
\setlist[itemize*]{itemsep=3pt, parsep=3pt, label=$\bullet$}
\setlist[itemize*,2]{label={--}}

\usepackage{parskip}
\sloppy
\widowpenalty=10000
\clubpenalty=10000

\newcommand{\cmddemo}[1]{\bigskip\parbox[c]{3.9cm}{\cmd{#1}}\parbox[c]{10cm}{\csname #1\endcsname}\bigskip}

\begin{document}

\TULfancytitlepage{Strojové vidění}
{Kevin Daněk}
{Liberec 2025}

\section{Strojové vidění}
Strojové vidění je název pro proces, ve kterém se získávají, zpracovávají a vyhodnocují digitální obrazy.

\section{Digitální obraz}
Představte si digitální obraz jako obrovskou mřížku, téměř jako tabulku. Tahle tabulka je tvořena malými barevnými body, kterým se říká pixely. Této tabulce obrazových bodů říkáme \textbf{matice pixelů}.

Ačkoliv nazýváme digitální obraz maticí obrazových bodů, je tento název poněkud zavádějící, protože se skutečnou \textit{maticí} mají společnou pouze podobu. Pokud máte za sebou kurz lineární algebry, víte, že matice představuje lineární zobrazení, a jsou pro ní definovány určité operace a algoritmy. Všichni se asi shodneme, že provádět \textit{Gaussovu}, potažmo \textit{Gauss-Jordanovu eliminaci} na obrázku asi žádné ovoce nepřinese. Proto se na digitální obraz většinou díváme spíše jako na funkci — konkrétně na tzv. \textbf{diskrétní obrazovou funkci}.



\section{title}

\end{document}

