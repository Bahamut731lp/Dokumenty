\documentclass[FM]{tulpresentation}

\title{Úvod do strojového vidění}
\subtitle{Přednášky na téma strojového vidění}
\author{Bc. Kevin Daněk}

\begin{document}

\TULtitleframe

\begin{frame}
	\frametitle{Strojové vidění}
	\begin{items}
		\item Zabývá se zpracováním obrazu
		\item Převádíme obrazová data na užitečnou informaci
		\item 
	\end{items}
\end{frame}

\begin{frame}
	\frametitle{Úlohy strojového vidění}
	\begin{items}
		\item Obecně můžeme analýzu digitálního obrazu rozdělit na následující úlohy
		\begin{subitems}
			\item \textbf{Klasifikace} - Roztřídění obrázků do tříd
			\item \textbf{Detekce} - Určení přítomnosti zájmového prvku
			\item \textbf{Segmentace} - Výřez zájmového prvku
		\end{subitems}
	\end{items}
\end{frame}

\begin{frame}
	\frametitle{Úlohy strojového vidění}
	\textit{TODO: Obrázek}
\end{frame}

\begin{frame}
	\frametitle{Barevný model}
	\begin{items}
		\item \textbf{Barevný model} určuje, jak se v obraze reprezentují barvy.
		\item Určuje počet barevných kanálů.
		\item Určuje povolené hodnoty pixelů v barevných kanálech.
	\end{items}
\end{frame}

\begin{frame}
	\frametitle{Jak vyjádřit barevný model}
	\begin{items}
		\item \textbf{Barevný model} určuje, jak se v obraze reprezentují barvy.
		\item Určuje počet barevných kanálů.
		\item Určuje povolené hodnoty pixelů v barevných kanálech.
	\end{items}
\end{frame}
\TULendframe
\end{document}
